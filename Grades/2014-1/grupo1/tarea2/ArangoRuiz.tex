\documentclass{article}
\usepackage[margin=1.0in]{geometry}
\usepackage{graphicx}


\begin{document}
\title{Biologia Celular}
\maketitle

\begin{Large}
\section*{La Celula}
\end{Large}

\subsection*{REL-RER}
       El reticulo endoplasmatico liso (RER) se encarga de sintetizar y modificar algunas proteinas, generalmente su ubicacion en la celula es despues del (RER) hacia la membrana.

El reticulo endoplasmatico rugoso (RER) tambien se encarga de sintetizar proteinas, además de esto mantiene una relacion con los ribosomas proveyendolos de una membrana para lograr su transpaso al aparato de golgi y otros organelos de la celula.

\subsection*{Aparato de Golgi}
Recoje y modifica proteinas que necesitan transformacion

\subsection*{Lisosomas}
Vesiculas que se generan en el golgi, tienen enzimas digestivas, y digieren componentes celulares ya utilizados.

\subsection*{Mitocondria}

Convierte en energia quimica potencial de la celula en energia (ATP) a travez de un proceso llamado respiración celular en donde entran nutrientes y oxigeno y producen ATP.
\begin{equation}
NUTRIENTES + O_2 = ATP
\end{equation}

\subsection*{Plastidio}              
Estan en celulas vegetales y algunas protistas                                 
\includegraphics{plastidios.png}

\subsection*{Teoria del Endosimbionte}
Las celulas eucariotas se originaroncuando una celula primitiva engolfo para alimentarse y la engolfada no se degrado, luego el organelo perdio mucho material genetico que le permitia vivir en contacto directo con el ambiente entpnces se completo la simbiosis.

\subsubsection*{Evidencia}
Se ha comprobado tranferencia de ADN entre el organelo y el nucleo además los cloroplastos cumplen funciones de bacterias fotosinteticas.


\end{document}