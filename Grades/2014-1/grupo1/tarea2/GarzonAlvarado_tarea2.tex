 \documentclass[12pt]{article}
\usepackage[margin=1.0in]{geometry}
\usepackage{amsmath}
\usepackage{textcomp}
\title{\LaTeX\ \textsc{Ejercicio geoquimica}}
\author{Juan Sebastian Garzon Alvarado}
\begin{document}
 
\maketitle
\date{}
\section{Meteorito}

Conocemos que la ecuacion experimental que relaciona el diametro $\Phi$ en Km con la energia potencial E en kilotones es :

\begin{equation}\label{eq:1}
\Phi =0,1 \sqrt[3]{E}
\end{equation}

Y sabemos que

\begin{equation}\label{eq:2}
\frac{1}{2} mV^2 = E
\end{equation}

Al tener que la densidad $\rho$ del meteorito esferico es de 6 $\frac{g}{cm^3}$ , su velocidad de entrada v es igual a 30 $\frac{Km}{s}$ y el diametro $\Phi$ es igual a 5 Km podemos hallar la masa y el diametro del meteorito.

\subsection{Masa meteorito}

De la ecuacion (1) obtenemos que :

\begin{equation}\label{eq:3}
(\dfrac{\Phi}{0,1})^3 = E
\end{equation}

De la acuacion (2) obtenemos que:

\begin{equation}\label{eq:4}
\dfrac{2E}{V^2} = m
\end{equation}
Al reemplazar (1) por los datos dados obtenemos que la energia expresada en kilotones es:
\begin{center}
 E=1250000 KT
\end{center}

Para lograr despejar la masa en (2) es necesario hacer la conversion de unidades de la velocidad de $\frac{Km}{s}$ a $\frac{m}{s}$ , adicionalmente debemos ajustar las unidades de Kilotones a Julios cuya relacion indica que:

\begin{center}
$4.184x10^{12} J$ = 1KT
\end{center}


Con estas conversiones obtenemos que la masa del meteorito es:
\begin{center}
m=$1.16x10^{12}$ Kg
\end{center}

\subsection{Diametro meteorito}

Para hallar el diamentro d, tenemos en cuenta las ecuaciones

\begin{equation}\label{eq:5}
\dfrac{m}{v} = \rho
\end{equation}

\begin{equation}\label{eq:6}
\dfrac{4}{3}\Pi r^3 = v
\end{equation}

\begin{equation}\label{eq:7}
 2r=d
\end{equation}

Con (5) y (6) obtenemos

\begin{center}
$ r=\sqrt[3]{\dfrac{3\dfrac{m}{\rho}}{4\Pi}} $
\end{center}

Al operar obtenemos que \\

\begin{center}
r= 3589.35 cm o r= 35.89m \\
\end{center}

Ya que el diametro del asteroide esferico es 2r obtenemos \\

\begin{center}
2(35.89m)= d \\


71.78m= d    \\

\end{center}


\begin{large}

 \textsl{Es importante resaltar que se ignoraron las cifras significativas de cada caso pues los datos entregados no reflejan la precision real de las mediciones.} 

\end{large}



\end{document}

