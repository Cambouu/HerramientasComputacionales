\documentclass{article}
\usepackage[margin=1.0in]{geometry}       
\usepackage{amsmath}                                                                                                                                                                                                                                                                                                                                                            
\title{Movimiento con aceleracion constante }%\latex}

\begin{document}
\maketitle


\section{Ecuaciones}
Es el movimiento en el que la velocidad cambia a ritmo constante con el tiempo.Cuando la aceleracion media para cualquier intervalo es la aceleracion.
Para encontrar una expresion para la velocidad primero sustituimos aceleracion media por aceleracion:

\begin{equation}
 a_{x}= \frac {v_{2X} - v_{1X}}{ t_{2} - t_{1}}
\end{equation}

Sea t$1$=0 y t$2$ cualquier instante posterior t. Simbolizamo von v$0x$ la componete x de la velocidad en el instante inicial t=0; la componente de la velocidad en el instante posterior t es v$x$. Entonces la ecuacion a se convierte en
\begin{equation}
 a_{x}= \frac {v_{x}-v_{0x}}{t-0}
\end{equation}

Solo con aceleracion constante:
\begin{equation}
 v_{x}= v_{0x} + a_{x}t
\end{equation}

Ahora deducieremos una ecuacion para la posicion x en funcion del tiempo cuando la aceleracion es constante. Para ello, usaremos
dos expresiones distintas para la velocidad media a$med-x$ en el intervalo de t=0 a cualquier t posterior. \\
La primera proviene de la definicion de v$med-x$, que se cumple sea constante o no la aceleracion. La posicion inicial es la posicion en t=0, denotada
con x$0$. La posicion en el t posterior es simplemente x. Asi, para el intervalo 

\begin{equation}
 \bigtriangleup{t} = t-0
\end{equation}
y el desplazamiento
\begin{equation}
 \bigtriangleup{x}= x- x_{0}
\end{equation}

La ecuacion da:
\begin{equation}
 v_{med-x} =\frac {x-x_{0}}{t}
\end{equation}
Tambien podemos obetener otra expresion para v$med-x$ que sea valida solo si la aceleracion es constante, de mnodo que la grafica v$x$-t sea una linea recta
y la velocidad cambie a ritmo constante. En este caso, la velocidad media en cualquier intervalo es solo el promedio de las velocidades al principio y al final del intervalo. 
Para el intervalo de 0 a t,
\begin{equation}
 v_{med-x} =\frac {v_{ox}+v_{x}}{2}
\end{equation}

Tambien sabemos que con aceleracion constante, la velocidad v$x$ en un instante t esta dada por la ecuacion (3). Sustituyendo esa expresion por v$x$ en la ecuacion
(7), solo con aceleracion constante:
\begin{equation}
 v_{med-x} =(v_{0x}+ v_{0x} + a_{x}t)\div2 = v_{0x} + a_{x}t\div{2}
\end{equation}
Por ultimo, igualamos las ecuaciones (6) y (8) y simplificamos el resultado:
\begin{equation}
 \frac {x-x_{0}}{t} = v_{0x} + a_{x}t\div{2}
\end{equation}
o
\begin{equation}
 x = x_{0} + v_{0x}t + a_{x}t\wedge{2}\div{2}
\end{equation}

Esta ecuacion (10) indica que si en el instante t=0, una particula esta en x$0$ y tiene velocidad v$0x$, su nueva posicion x en un t posterior es la suma de tres terminos: su posicion inicial
x$0$, mas la distancia v$0$t que recorreria si su velocidad fuera constante, y una distancia adicional 
\begin{equation}
 (a_{x}t\wedge{2})\div{2} 
\end{equation}
causada por el cambio develocidad.
Las ecuaciones anteriores se pueden comprobar con el supuesto de aceleracion constante derivando la ecuacion (10). Obtenemos:
\begin{equation}
 v_{x}= \frac{dx}{dt}= v_{0x}+ a_{x}t
\end{equation}
que es la ecuacion (10). Diferenciando otra vez, tenemos simplemente
\begin{equation}
\frac{dv_{x}}{dt}= a_{x}
\end{equation}
que concuerda con la deficion de aceleracion instantanea.
Con frecuencia es util tener una relacion entre posicion, velocidad y aceleracion que no incluya el tiempo. Para obetenerla, despejamos t en la ecuacion 
(3), sustituimos la expresion resultante en la ecuacion (10) y simplificamos:
\begin{equation}
t= \frac{v_{x}-v_{0x}}{a_{x}}
\end{equation}
\begin{equation}
 x = x_{0} + v_{0x}(\frac{v_{x}-v_{0x}}{a_{x}}) + a_{x}(\frac{v_{x}-v_{0x}}{a_{x}})\wedge2\div{2}
\end{equation}
Trasnferimos el termino x$0$ al miembro izquierdo y multiplicamos la ecuacion por 2a$x$:
\begin{equation}
2a_{x}(x-x_{0})= 2v_{0x}v_{x} -2(v_{ox})\wedge{2} + v_{x}\wedge{2} -2v_{0x}v_{x} + v_{0}\wedge{2}
\end{equation}
Por ultimo, al simplificar obtenemos:
\begin{equation}
v_{x}\wedge{2}= v_{0x}\wedge{2} +2a_{x}(x-x_{0})
\end{equation}
Para obtener una relacion mas util igualando dos expresiones para v$med-x$, y multimplicando por t. Al hacerlo, obtenemos: 
\begin{equation}
x-x_{0} = \frac{(v_{0x}+v_{x})t}{2}
\end{equation}
Observe que la ecuacion (18) no contiene la aceleracion a$x$. Esta ecaucion es util cuando a$x$ es constante pero se desconoce su valor.
Las ecuaciones (3),(10), (17) y (18) son las ecuaciones del movimiento con aceleracion constante. Con ellas, podemos resulver cualquier problema que implique movimiento 
rectilineo de una particula con aceleracion constante.
\\En el caso especifico de movimiento con aceleracion constante, los valores de x$0$, v$0x$ y a$x$ son positivos. Un caso especial de movimiento con aceleracion constante 
se da cuando la aceleracion es cero.
La velocidad es entonces constante, y las ecuaciones del movimiento se convierten sencillamente en:
\begin{equation}
v_{x}= v_{0x}=constante
\end{equation}
\begin{equation}
x = x_{0} + v_{x}t
\end{equation}
\end{document}
