\documentclass{article}
\usepackage[margin=1.0in]{geometry}
\usepackage{amsmath}

\title{clase}


\begin{document}

\date{7/2/2014}


\maketitle


\section{Qumica}


presicion = reproductibilidad//

Exactitud: que tanto se acerca la medida a un valor verdadero

\subsection{Atomos, moleculas y iones}
 

Numero Atomico (z): numero de protones en un atomo //
Numero de masa (A): numero de protones + neutrones //
Isotopos: atomos de un mismo elemento que difieren en el numero de neutrones //
(tienen igual Z y Diferente A)//

               Fe(54)//
               Fe(56)//
               Fe(57)//
               Fe(58)//

\subsection{-Compuestos Ionicos}

combinaciones de metales y no metales

\subsection{-Acidos}

tiene protones (H) para ceder

\section{.Ecuaciones}



\begin{equation} 



XNeutrones : \dfrac{#neutrones}{total particulas}

//

%elemento= \dfrac{#Atomos x peso elemento}{peso total compuesto}


\end{equation}









\end{document}

