\documentclass[12pt]{article}
\usepackage{amsmath}
\usepackage[margin=2.0in]{geometry}
\title{Notas Fisica 1}

\begin{document}
\maketitle
\section*{Leyes de newton}
1. Un cuerpo permanece estatico o en movimiento rectilineo uniforme a menos que una fuerza cambie su trayectoria.\\
2. El cambio de movimiento en un cuerpo es proporcional a la fuerza que se le aplica.\\
3. Cuando un cuerpo ejerce una fuerza este recibe una fuerza de igual magnitud y direccion opuesta.\\
1.
\begin{equation}
 \sum\vec{F}=0
\end{equation}
2.
\begin{equation}
 \sum\vec{F}=m\vec{a}
\end{equation}
3.
\begin{equation}
 \sum\vec{F}_{a en b}=Σ\vec{F}_{b en a} 
\end{equation}
\section*{Trabajo y Energia}
\subsection*{Trabajo}
\begin{equation}
 W=n*m=J
\end{equation}
El trabajo es igual al producto punto entre los vectores fuerza y desplazamiento
\begin{equation}
 \vec{F}\cdot{\vec{D}}=\vec{F}*\vec{D}*cos(Φ)
\end{equation}
Trabajo expresado en forma de integral
\begin{equation}
 W=\int F(x)dx
\end{equation}
\subsection*{Energia cinetica}
\begin{equation}
 K=\frac{1}{2}mV^{2}
\end{equation}
\subsection*{Energia potencial elastica}
\begin{equation}
 U_{e}=\frac{1}{2}kx^{2}
\end{equation}
\subsection*{Energia potencial gravitacional}
\begin{equation}
 U_{g}=\frac{1}{2}
\end{equation}
\subsection*{Conservacion de la energia}
\begin{equation}
 K_{i}+U_{i}+W=K_{f}+U_{f}
\end{equation}

\end{document}
