\documentclass{article}
\usepackage[margin=3cm]{geometry}
\usepackage{amsmath}
\title{Notas Ecuaciones Diferenciales}

\begin{document}
\date{Febrero7 de 2014}
\maketitle
Nataly Catillo 
Cod.  201126924
 
\section{Ecuacion de Rikati}

Si se tiene una ecuacion de la forma
\begin{equation}
  \frac{dy}{dx}= a_{1}(t)+a_{2}(t)y+a_{3}(t)y^{2}
\end{equation}
condiciones:

 y es una variable dependiente que NO aparece explicitamnete 


 t es una variable independiente que NO aparece explicitamente 

y posee una soluci\'on particular de (1),  y_{1} 

 Entonces: 

\begin{equation}
 y=y_{1}+\frac{1}{v_{(t)}}
\end{equation}

\begin{equation}
  \dot{y}=\dot{y_{1}}-\frac{\dot{v}_{t}}{v^{2}} 
\end{equation}

si

 
\begin{equation}
 \dot{y_{1}}-\frac{\dot{v_{t}}}{v^{2}}=a_{1}(t)+a_{2}(t)[y_{1}+\frac{1}{v_{(t)}}]+a_{3}(t)[y^{2}_{1}]
\end{equation}


Sabemos que  \ a_{1}(t),\ a_{2}(t) y \ a_{3}(t) \ son iguales  a  \ \dot{y_{1}}\ ya  que este ultimo es solucion entonces 


\begin{equation}
 -\frac{\dot{v_{t}}}{v^{2}}=\frac{a_{2}(t)}{v_{(t)}}+\frac{2a_{3}(t)y_{1}}{v_{(t)}}+\frac{a_{3}(t)}{v^{2}}
\end{equation}

Entonces 

\begin{equation}
 \dot{v_{t}}=-v(a_{2}(t)+2a_{3}(t)y_{1})-a_{3}(t)
\end{equation}


La anterior es una ecuacion diferencial lineal de primer orden que se  puede  resolver por los  m\'etodos  vistos en clase. Esta formula se  puede utilizar siempre y cuando la solucion particular sea evidente o fasil de calcular.


\section{reduccion de orden}


Si y[variable dependiente] No aparece explicitamente. Entoces se puede hacer  un cambio de variable de la forma:

\begin{equation}
 v=\dot{y_{(t)}} , 
\dot{v}=\ddot{y_{(t)}}
\end{equation}

Y si t[variable dependiente] que No aparece explicitamente 

\begin{equation}
 \dot{y}=v,
 \ddot{y}=\frac{dv}{dy}
\end{equation}

\begin{equation}
 \frac{dy}{dt}=v\frac{dv}{dy}
\end{equation}


entonces:

si 
\begin{equation}
 A_{(t)}\ddot{y}+B_{(t)}\dot{y}+C_{te}=0
\end{equation}

\begin{equation}
 A_{(t)}\dot{R}+B_{(t)}{R}+C_{te}=0
\end{equation}


\section*{Ejemplo}

si  se  tiene la siguiente ecuacion de segundo orden

\begin{equation}
 t^{2}\ddot{y}+t\dot{y}-1=0
\end{equation}

se puede convertir en primer orden  haciendo la sustituci\'on (7) y  convertirla en:

\begin{equation}
 t^{2}\dot{y}+t{y}=1
\end{equation}
 
si la simplificamos 

\begin{equation}
 \dot{v}+ \frac{v}{t}=\frac{1}{t^{2}}
\end{equation}

La  ecuaci\'on (14) se  puede  resolver por  factor integrante. Donde: 
\[
\mu=e^{\int\frac{1}{t}dt}=t
 \]

se  multiplica (14) por \mu\  y se resuelve, teniendo 

\begin{equation}
 tv=\int\frac{1}{t}dt=\ln(t)+C_{te}
\end{equation}
\[
 v=\frac{\ln(t)}{t}+\frac{C_{te}}{t}
\]

Como sabemos que por la sustitucion se cumple (7). Entonces

\begin{equation}
 y=\int\frac{\ln(t)}{t}+\frac{C_{te}}{t}dt
\end{equation}
\[
 y=\int\frac{\ln(t)}{t}dt + \int\frac{C_{te}}{t}dt
\]
\[
 y=\frac{1}{2}\ls(t^{2})+C\ln(t)
\]

\end{document}
