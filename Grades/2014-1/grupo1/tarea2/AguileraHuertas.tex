\documentclass[12pt]{article}
\usepackage{amsmath}
\usepackage[utf8]{inputenc}
\usepackage[T1]{fontenc}
\usepackage{lmodern}
\usepackage[spanish]{babel}

\title{ Notas Vectorial-Derivadas Parciales}

\begin{document}
\maketitle


\section{Definición}\\ \\

\varinjlim $\dfrac{\textflorin(Xo + h)-\textflorin(Xo,Yo)}{h}$ \\ \\

La derivada parcial de la funcion, es en relaciòn,la pendiente de la curva formada por la coordenada tangente.\\ \\

F{(x,y) =$x^2y$\\ \\


$\dfrac{{\partial f}}{{\partial x}}(1,1)\\ \\

\varinjlim $\dfrac{(1+h)^2-1}{h}$ \\ \\


\varinjlim $\dfrac{h^2+2h}{h}$\\ \\

\varinjlim $(h+2) = 2 \\ \\

\checkmark $x^2 -y^2$\\ \\

$\dfrac{{\partial f}}{{\partial x}}$(1,1) = -2\\ \\

$\dfrac{{\partial x}}{{\partial x}}(1,1) = -2\\ \\


$\dfrac{{\partial (xe^x+(\tan(h^2)(y))}}{{\partial x}}$\\ \\

$\dfrac{{\partial (xe^xe^{(\tan(h^2)(y)})}}{{\partial x}}$\\ \\

${e^{\tan(h^2y)}}{\dfrac{\partial(xe^2)}{\partial x}$\\ \\

${e^{\tan(h^2y)}}{(xe^x+e^x)}$\\ \\

\varinjlim ${\dfrac{F(x,y)-(F(X0,Y0)+{\dfrac{\partial F(X0,Y0)(X-X0)}{\partial x}}+{\dfrac{\partial F(X0,Y0)(Y-Y0)}{\partial Y}}}{\sqrt{(X-X0)^2+(Y-Y0)^2}}}$ \\ \\







Si es diferenciable, existe una transformaciòn lineal que aproxima a ese punto\\ \\


Diferenciable: En 2 puntos de un plano, es decir, de varias variables.\\ \\



 

 

\end{document}
