\documentclass[12pt]{article}
\usepackage{amsmath}
\usepackage[margin=1.0in]{geometry}
\usepackage{textcomp}


\title{Mec\'anica del continuo}
\author{Lina P\'erez \'Angel 201025904}


\begin{document}


\maketitle
\textbf {Fricci\'on y Esfuerzo}\\


Existen dos tipos de tensi\'on: \\
Tensi\'on \textit{Superficial:}\\
\begin{equation}\label{eq:1.1}
\dfrac{\vec{F}}{\vec{A}}
\end{equation}
la cual es un tensor de grado 2\textdegree \\



Tensi\'on \textit{Cuerpo:}\\
\begin{equation}\label{eq:1.2}
\dfrac{\vec{F}}{m}
\end{equation}
la cual es un tensor de grado 1\textdegree \\



Fricci\'on \textrightarrow tensi\'on cortante.\\

Sin punto de contacto no puede haber tensi\'on cortante\\

\textsl{Shear} Stress + \textbf{Normal} Stress \textrightarrow no puedo tener fricci\'on sin normal. 



\begin{large}
 \section {\textbf{Ley de AMONTON} }
\end{large}

\begin{equation}
 \dfrac{Ff}{N} = constante (\textit{f}) = tan \phi
\end{equation}

se realiza un esquema con diagramas de fuerza, plano inclinado. Hacemos sumatoria de fuerzas en X y en Y.\\

\begin{large}
 \textup{Sumatoria de X}
\end{large}

\begin{equation}
 \sum F_{x} = Ffcos\alpha - Nsen\alpha = 0
\end{equation}

\begin{large}
 \textup{Sumatoria de Y}
\end{large}
\begin{equation}
 \sum F_{y} = Ncos\alpha + Ffsen\alpha - w = 0
\end{equation}

Es decir que:\\
\begin{equation}
 Ff = \dfrac{sen\alpha}{cos\alpha} N = Tan\alpha N
\end{equation}

$\alpha$ es el \'angulo de deslizamiento, si este es mayor a $\phi$ se desliza, y si $\alpha$ es menor a $\phi$  se queda quieto. \\
\textrightarrow La rugosidad de la superficie genera mayor o menor fricci\'on.\\

\begin{large}
 \subsection{ Sumatoria de fuerzas para dos rocas sedimentarias las cuales esten clasto-sedimentadas. }
\end{large}

Siempre voy a tener en cuenta la gravedad, la normal con la sueprficie de contacto y la dirección de la feurza del fluido que est\'e pasando por al roca. Por ejemplo, el agua de un rio. \\

\begin{large}
 \textup{Sumatoria de X}
\end{large}
\begin{equation}
 \sum F_{x} = F - Nsen\alpha = 0
\end{equation}


\begin{large}
 \textup{Sumatoria de Y}
\end{large}
\begin{equation}
  \sum F_{y} = Ncos\alpha - mg = 0
\end{equation}


Los poros de la roca se aumentan cuando le aplico una presi\'on a la arenisca. Llega a un punto m\'aximo donde se empieza a reducir. \\

\begin{large}
 \subsection{ Qu\'e pasa para que un granito se sostenga con una pendiente tan alta? }
\end{large}
 
 el angulo de fluencia \textrightarrow activacion del deslizamiento . si $\alpha$ es mayor que  $\phi$f\\
 el angulo de cortante residual \textrightarrow deslizamiento se detiene\\

\begin{equation}
 Presion = \dfrac{\vec{F}}{\vec{A}} = S
\end{equation}

\begin{equation}
 \vec{F}= \vec{A} * S = \c{k} N S 
\end{equation}

\begin{equation}
 \dfrac{\vec{F}}{N} = \c{k} S
\end{equation}

esto va a determinar la fricci\'on interna de la roca. \\
ya que el \'Area va a ser proporcional a la fuerza normal. tenemos que\\

\begin{equation}
 \c{k}N = \vec{A}
\end{equation}





\end{document}
