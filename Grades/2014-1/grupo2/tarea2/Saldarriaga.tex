\documentclass[12pt]{article}
\usepackage[margin=1.0in]{geometry}
\usepackage[spanish]{babel}
\usepackage{amsmath}
\author{Santiago Saldarriaga}


\title{Clase Anadec}

\begin{document}

\maketitle

\section{Intereses}
Existen diferentes tipos de intereses para las entidades prestamistas y financieras
\begin{itemize}
 \item{1)} Simple/Compuesto
 \item{2)} Nominal /Efectivo los cuales deben ser:Vencido/anticipado
 \item{3)} Corriente/Constante
 \item{4)} Capitalizacion discreta/ Capitalizacion continua
\end{itemize}
\subsection{Interes Nominal}
\begin{itemize}
 \item Usado para la liquidacion comercial de operaciones de credito donde el interes se paga periodicamnete
 \item Normalmente se expresa en terminos anuales (nominal anual- NA) y se indica la periodicidad de capitalizacion de los intereses (periodos distintos a un año)
 \item Periodicidad de pago = vencido o anticipado
\end{itemize}
\subsubsection{Ejemplo:}
Se desea abrir un fondo de inversion. Existen dos alternativas de intereses:
\begin{itemize}
 \item{1)} 12 porciento NA/SV
 \item{2)} 12 porciento NA/MV
\end{itemize}
\subsubsection{Solucion:}
Para resolver esto, se le debe quitar el nombre al interes con la siguiente formula
\begin{equation}
 \dfrac {Interes}{Periodo} \equiv Interes por periodo  
\end{equation}

\begin{equation}
 I(ea)=(1+Interespor periodo)^2-1
\end{equation}

\begin{itemize}
 \item{1)} 
Aqui el interes por semestre vencido es 6 porciento. Ahora se busca encontrar el interes efectivo anual (ea):
\begin{equation}
 I(ea)=(1+0.06)^2-1=12.36 i(ea)
\end{equation}
  \item{2)} 
Aqui el interes por mes vencido es 1 porciento. Ahora se busca encontrar el interes efectivo anual (ea):
\begin{equation}
 I(ea)=(1+0.01)^2-1=12.68 i(ea)
\end{equation}
\end{itemize}
De este ejemplom se puede concluir que a mayor frecuencia de pago, mayor valor acumuladon de interes.
\subsubsection{Problemas con el interes nominal}
Aunque es muy usado, este tiene las siguientes desventajas:
\begin{itemize}
 \item No captura directamente el valor del dinero en el tiempo
 \item No permite la comparacion directa de operaciones financieras
 \item Siempre requiere el periodo de pago y la forma de pago.

\end{itemize}

\section{Plana para calcular el interes efectivo anual:}
\begin{itemize}
 \item{1)} Si encuentro NA/Periodo o NS/Periodo
 \item{2)} Se le quita el nombre al interes dependiendo del numero de periodos que tiene el interes
 \item{3)} Dividir por el numero de periodos para obtener el interes vencido
 \item{4)} Usar la formula para el interes efectivo anual:
\begin{equation}
 (1+ interes-efectivo)=(1+Interes-vencido)^2
\end{equation}
\end{itemize}

\end{document}
