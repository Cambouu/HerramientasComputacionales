\documentclass {article}
\usepackage[margin=1.0in]{geometry}
\usepackage[utf8]{inputenc}
\usepackage{lmodern}
\usepackage[spanish]{babel}
\usepackage{amsmath}
\author{Felipe Berón}
\title{Ingeniería de Materiales}
\begin{document}
\maketitle
\section*{Temas}
\subsection*{Metales}
Tratratamientos termicos
\begin{itemize}
 \item Endurecedores
 \item Suavizadores
 \item Volumterica
 \item Superficial
\end{itemize}
\subsection*{Cerámicos}
\begin{itemize}
 \item Weibull
 \item Cristalinos
 \item Amorfos + cristalinos
\end{itemize}
\subsection*{Polímeros}
\begin{itemize}
 \item Amorfos
 \item Semicristalinos
 \item Viscoelasticidad (t, T)
\end{itemize}
\section*{Calificación}
\begin{itemize}
 \item3 Parciales 75$\%$
 \item Tallere, tareas, quices, experimentos, laboratorios, proeyecto* 25$\%$.
\end{itemize}
*No todos valen lo mismo
\section*{Introducción Metales}
\subsection*{Contexto}
\begin{itemize}
 \item História
 \item Particularidad Colombia
 \item Mundo
\begin{itemize}
 \item Producción
 \item Mercado
\end{itemize}
\end{itemize}
Encontrados semipuros en la naturaleza
\begin{itemize}
 \item Au, Ag, Cu, Pt
\end{itemize}
Los demas: Oxidos
\begin{itemize}
 \item Piedra (oxidos) - Fe
\end{itemize}
Abundancia en la corteza terrestre
\begin{itemize}
 \item Al = 8,07 $\%$
 \item Fe = 5,05 $\%$
\end{itemize}
Producción Fe y Acero = 95,45 $\%$\\
A partir del horno de Bessemer se dispara la producción de aceros\\
Actualmente $\simeq$ 1400 Millones de Toneladas (MTM)
\begin{itemize}
 \item Aluminio $\simeq$ 40 MTM
 \item Cobre $\simeq$ 17 MTM
 \item Zinc $\simeq$
 \item Plomo $\simeq$ 8 MTM
 \item Metalurgias de polvo, sinterización. Escenario especializado.
\end{itemize}
\subsection*{Tipos de Hierro}
\begin{itemize}
 \item Hierro fundido $\%$C $>$ 2
 \item Hierro gris = ojuelas cruzadas de grafito
 \item Hierro dúctil = grafito en esféras
 \item Hierro maleable = cememntita semiesférica
\end{itemize}
Aceros bajas aleaciones, plano= recocido\\
Alta resistencia =tratamientos térmicos.
\subsection*{Tipos de Hornos}
\begin{itemize}
 \item Acido: solo ajusta Magnesio y Silicio
 \item Básico: ajusta Fósforo, Manganeso, Sulfuro y Silicio
\end{itemize}

\end{document}
