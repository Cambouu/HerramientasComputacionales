\documentclass{article}
\usepackage[margin=1.0in]{geometry}
\usepackage[utf8]{inputenc}
\usepackage[T1]{fontenc}
\usepackage{lmodern}
\usepackage[spanish]{babel}
\usepackage{amsmath}
\title{CLASE MECANICA DE FLUIDOS - HIDROSTÁTICA}

\begin{document}
\date{03 de Febrero de 2014}
\maketitle 
\subsection{Manometria}

El metodo utiliza el cambio de presiòn con la elevaciòn
\subsubsection{Piezòmetrico}
Tambien llamado manòmetro simple
\begin{equation}
 Pa = {\gamma}h
\end{equation}
este tiene limitaciones: presiones altas y flujo de gas...
\subsubsection{Manòmetro Diferencial}
o tubo en U 
\begin{equation}
 P2 = 0 +{\gamma}{\varDelta}h
\end{equation}
\begin{equation}
 P2 = P3
\end{equation}
\begin{equation}
 P3 = P4 + {\gamma}l
\end{equation}
en un Liquido manòmetrico $ {\gamma}m > \gamma  $

Ejemplo: 
para un manometro en u, la presion en un punto Pa es igual a: 
\begin{equation}
 Pa = Patm + {\gamma}Hg{\varDelta}H - {\gamma}H2O{\varDelta}H - {\gamma}H2O * H  
\end{equation}

\subsection{Medicion de la presiòn - Flexiòn de membranas elàsticas y transductores de presiòn}
\subsubsection{Medidor Tubo De Bourdon }
tubo de secciòn transversar elìptica doblado en forma de arco circular. Cuando se aplica presión el tubo curvo tiende a enderezarse actuando sobre una aguja de medición
\subsubsection{Transductores de presiòn}
Diseñados para producir señales electronicas que pueden ser transmitidas a osciloscopios o aparatos digitales para almacenamiento de registros y  controlar otros apartatos para operaciòn de procesos

De diafragma flexible conectado a un sensor. Ejemplo: Cable defromable de resitencia variable (deformimetro). Al cambiar la longitud cambia la resistencia del cable lo cual es utilizado electronicamente para producir un cambio de voltaje.

Se utilizan sistemas de adquisiciòn de datos con los transductores de presión. La señal analóga del transductor se convierte a señal digital. 

\subsection{Fuerzas hidrostáticas sobre superficies planas}
Son superficies HORIZONTALES inmeresa en un liquido o expuestas a la presión uniforme sobre la superficie.

La fuerza total resultante es igual al producto de la presión y el área de la superficie.

La fuerza resultante actúa sobre el centroide del área y su linea de acción es normal al área.

\subsubsection{Centroide}

\begin{equation}
 UAT= (U1A1 + U2A2)
\end{equation}

\begin{equation}
 u=\dfrac {(U1A1 + U2A2)}{A1 + A2}
\end{equation}

 Sobre una superficie plana inclinada la presión se distribuye linealmente, Se utiliza un preocedimiento general para evaluar la magnitud de la fuerza resultante y la localización de su línea de acción.

\subsubsection{Magnitud de la fuerza resultante}
Considere una superficie plana inclinada un ángulo $\alpha$ con la horizontal, sumergida en agua

la fuerza total sobre el área es: 

\begin{equation}
 F = \int p dA  = \int{\gamma} y {\sin}{\alpha}dA
\end{equation}

La magnitud de la resultante de la fuerza hidrostática sobre un área plana es el producto de la presión en el centroide del área y el área de dicha superficie

\begin{equation}
F = Pcentroide * A 
\end{equation}

\subsection{Localización vertical de acción de la resultante de la fuerza hidrostática}
En general, la localización de la línea de acción de la fuerza hidrostática resultante es abajo del centroide pues la presión aumenta con la profundidad.

Esta determinada por: 

\begin{equation}
Ycp = Ycentroide + \dfrac {I}{Ycentroide*A}
\end{equation}

\begin{equation}
Ycp - Ycentroide = \dfrac {I}{Ycentroide*A}
\end{equation}

\end{document}



