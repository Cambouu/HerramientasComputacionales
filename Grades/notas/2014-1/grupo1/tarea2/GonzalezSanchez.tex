\documentclass[twocolumn]{article}
\usepackage{amsmath}
\usepackage[margin=1.0in]{geometry}
\title{Matrices}

\begin{document}
\date{}
\maketitle
\section{Noci\'on de una matriz y \'algebra de matrices}

Una matriz es un arreglo rectangular de n\'umeros, por ejemplo:
$$
\begin{pmatrix}
1 & 4 & 6 \\
3 & 1 & -7 
\end{pmatrix}
$$
En general, una matriz frecuentemente se escribe as\'i:
$$
A = (a_{ij})_{mxn} =
\begin{pmatrix}
a_{11} && a_{12} && \cdots && a_{1n} \\
a_{21} && a_{22} && \cdots && a_{2n} \\
\vdots && \vdots && \ddots && \vdots \\
a_{m1} && a_{m2} && \cdots && a_{mn}
\end{pmatrix}
$$
y se dice que es una matriz de tamaño mxn, o que est\'a compuesta de $m$ filas y $n$
columnas. Por ejemplo, la primera fila es:
$$
A_{1n}= 
\begin{pmatrix}
a_{11} && a_{12} && \cdots && a_{1n}
\end{pmatrix}
$$
La segunda columna es:
$$
A_(m2)=
\begin{pmatrix}
a_{12} \\
a_{22} \\
\vdots \\
a_{m2} 
\end{pmatrix}
$$
En general denotaremos por $A_{in}$ la i-\'esima fil y por $A_{mj}$ la j-\'esima columna de
la matriz A. El elemento $a_{ij}$ se encuentra localizado en la intersecci\'on de la 
i-\'esima fila y la j-\'esima columna. 
$$
\begin{pmatrix}
 && \vdots && \\
 \cdots && a_{ij} && \cdots \\
 && \vdots && \\
\end{pmatrix}
$$
Definici\'on 1:

Una matriz A de tamaño nxn, es decir, cuando es igual el n\'umero de filas al de columnas,
se denomina una matriz cuadrada de orden n.

En una matriz cuadrada A los elementos $a_{ii}$ se denominan elementos diagonales.

Definici\'on 2:

Dos matrices $A = (a_{ij})_{mxn}$ y $B = (b_{ij})_{mxn}$ (del mismo tamaño) son iguales si
todos los elementos correspondientes son iguales, esto es, si
$$ a_{ij}=b_{ij} $$
para todos los $i=$ 1,2,...$m$ y todos los $j=1,2,...n$.

Definici\'on 3:

Sean $A=(a_{ij})_{mxn}$ y $B = (b_{ij})_{mxn}$.

La suma de esas dos matrices es la matriz $C = c_{ij})_{mxn}$ donde $c_{ij}=a_{ij}+b_{ij}$.
Esto es, la suma de dos matrices del mismo tamaño es la matriz de ese mismo tamaño, obtenida
al sumar los correspondientes elementos.

Definici\'on 4:

Una matriz en que todos sus elementos son ceros se denomina matriz nula (o matriz cero) y se
denota por 0.

Definici\'on 5:

Sea $c$ un escalar y $A = (a_{ij})_{mxn}$ una matriz. El producto $cA$ es la matriz $B = (b_{ij})_{mxn}$ donde $b_{ij}=ca_{ij}$.

Definici\'on 6:

Sean $A = (a_{ij})_{mxn}$ y $B = (b_{ij})_{mxn}$. Se define la diferencia A-B por
$$ A-B= A+(-1)B $$

Teorema 1:

Sean A, B, C matrices de tamaño $m x n$ y $c$ y $d$ escalares, entonces:

\begin{itemize}
	\item (A+B)+C=A+(B+C)
	\item A+B=B+A
	\item A+0=A (0 es la matriz nula)
	\item A+(-A)=0 (la matriz cero)
	\item c(A+B)=cA+cB
	\item (c+d)A=cA+dA
	\item (cd)A=c(dA)
	\item 1 A=A
\end{itemize}

Definici\'on 7:

El producto de una matriz 1xn (llamada vector fila) por una nx1 (llamada vector columna) se
define como:
\begin{equation}
\begin{pmatrix}
a_1 && a_2 && \cdots && a_n
\end{pmatrix}
\cdot
\begin{pmatrix}
b_1 \\
b_2 \\
\vdots \\
b_n
\end{pmatrix}
= a_1b_1+a_2b_2+\cdots+a_nb_n = \displaystyle\sum_{i=1}^n a_ib_i
\end{equation}
obs\'ervese que este producto es un n\'umero.

Definici\'on 8:

Sean $A = (a_{ij})_{mxn}$ y $B = (b_{ij})_{mxn}$. El producto AB es otra matriz $C = (c_{ik})_{mxn}$ donde 
$$
c_{ik}= 
\begin{pmatrix}
a_{i1} && a_{i2} && \cdots && a_{in}
\end{pmatrix}
\begin{pmatrix}
b_{1k} \\
b_{2k} \\
\vdots \\
b_{nk}
\end{pmatrix}
= \displaystyle\sum_{j=1}^n a_{ij}b_{jk}
$$

Luego, el elemento que se encuentra en la posici\'on $ik$ de AB se obtiene al multiplicar la 
fila $i$ de A por la columna $k$ de B. N\'otese que la multiplicaci\'on de una matriz A con
una matriz B solamente se define cuando el n\'umero de columnas de A es igual al n\'umero de
filas de B. Adem\'as, el tamaño de la matriz producto es:
\begin{center}
($\#$ de filas de A) X ($\#$ de columnas de B)
\end{center}

Definici\'on 9:

Una matriz cuadrada de orden $n$ que tiene todos los elementos diagonales iguales a 1 y 
todos los dem\'as componentes iguales a 0 se llama matriz identidad (o id\'entica) de orden
$n$ y se denota por $I_n$
$$ I_n=
\begin{pmatrix}
1 && 0 && 0 && \cdots && 0 \\
0 && 1 && 0 && \cdots && 0 \\
0 && 0 && 1 && \cdots && 0 \\
\vdots && \vdots && \vdots && \ddots && \vdots \\
0 && 0 && 0 && \cdots && 1 
\end{pmatrix}
$$

Teorema 2:

Al suponer que A, B, C son de tamaño tal que pueden efectuarse todas las operaciones
indicadas (se dice que las matrices son conformables indicadas) y que $c$ es un escalar;
entonces
\begin{itemize}
\item c(AB) = (cA)B = A(cB)
\item A(BC)=(AB)C
\item IA=A y BI=B
\item A(B+C)=AB+AC
\item (A+B)C=AC+BC
\end{itemize}

Demostraci\'on

i) Probemos primero que $c(AB)=(cA)B$. Sea $A = (a_{ij})_{mxn}$ $B = (b_{jk})_{nxp}$

El elemento que se encuentra en la posici\'on $ik$ en la matriz $c(AB)$ es
$$ c(\displaystyle\sum_{j=1}^n a_{ij}b_{jk})$$

El elemento que se encuentra en la posici\'on $ik$ en la matriz $(cA)B$ es
$$ \displaystyle\sum_{j=1}^n (ca_{ij})b_{jk} = c(\displaystyle\sum_{j=1}^n a_{ij}b_{jk}) $$

Luego por la definici\'on 2) de igualdad de matrices, se desprende que $$c(AB)=(cA)B$$

ii) Segundo, sea $A = (a_{ij})_{mxn}$, $B = (b_{jk})_{nxp}$, $C = (c_{ks})_{pxq}$

El elemento que se encuentra en la posici\'on $is$ en la matriz A(BC) se obtiene al multiplicar la fila i de A por la columna s de BC, por tanto, es
$$ \begin{pmatrix}
a_{i1} && a_{i2} && a_{in}
\end{pmatrix}
\begin{pmatrix}
\displaystyle\sum_{k=1}^p b_{1k}c_{ks} \\
\displaystyle\sum_{k=1}^p b_{2k}c_{ks} \\
\vdots \\
\displaystyle\sum_{k=1}^p b_{nk}c_{ks} 
\end{pmatrix} $$

$$ = a_{i1} \displaystyle\sum_{k=1}^p b_{1k}c_{ks} + a_{i2} \displaystyle\sum_{k=1}^p b_{2k}c_{ks} + \cdots + a_{in} \displaystyle\sum_{k=1}^p b_{nk}c_{ks}$$

$$ = \displaystyle\sum_{j=1}^n a_{ij} \displaystyle\sum_{k=1}^p b_{jk}c_{ks} $$

El elemento que se encuentra en la posici\'on $is$ en la matriz (AB)C se obtiene, al multiplicar la fila $i$ de AB por la columna s de C, esto es:
$$\displaystyle\sum_{k=1}^n (\displaystyle\sum_{j=1}^p a_{ij}b_{jk})c_{ks}$$
$$ = \displaystyle\sum_{j=1}^n a_{ij} \displaystyle\sum_{k=1}^p b_{jk}c_{ks} $$

Luego, por la definici\'on 2), se deprende que 
\begin{center}
(AB)C=A(BC)
\end{center}


\end{document}