\documentclass[12pt]{article}
\usepackage[margin=1.0in]{geometry}
\usepackage{amsmath}
\title{Laboratorio Ondas y fluidos}

\begin{document}
\date{}
\maketitle
\section*{Pre-informe 1: Circuito RC con osciloscopio}
\section{Carga del condensador}

\subsection*{carga}
La carga de de un condensador con capacitancia C, empieza en el momento que el interruptor del circuito se cierra y permite el paso de corriente. El condensador comienza a cargarse en el instante $ t = 0 $, este proceso esta descrito por la ecuacion:  

\begin{equation}
  q(t) = v_0C(1 - e^{\dfrac{-t}{RC}})$$

$$\end{equation}
 V : voltaje inicial\\
 C : Capacitancia del condensador\\
 VC : carga incial\\

\subsection*{corriente}

La corriente en el circuito esta definida por: 

\begin{equation}
 \dfrac{dq(t)}{dt}} = i(t) = ie^{\dfrac{-t}{RC}}$$

$$\end{equation}

\section{descarga del condensador}
\subsection*{carga}

En el proceso de descarga del condensador con caraga inicial $ q_0 $ y corriente $ i_0 = 0$, la carga del mismo esta descrita por la ecuacion :

\begin{equation}
q(t) = q_0e^{\dfrac{-t}{RC}}$$
$$\end{equation}

\subsection{corriente}

El comportamiento de la corriente mientras el condensador de descarga obdece a la derivada de la ecuacion de la carga:\\

\begin{equation} 
q_0e^{\dfrac{-t}{RC}}$$
 
$$\end{equation}

\begin{Huge}
 Juan Pablo Molano

\end{Huge}












\end{document}
