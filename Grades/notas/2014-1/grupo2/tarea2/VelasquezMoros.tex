\documentclass[12pt]{article}

\usepackage[margin=1.0in]{geometry}
\usepackage[utf8]{inputenc}
\usepackage[T1]{fontenc}
\usepackage{lmodern}
\usepackage[spanish]{babel}
\usepackage{amsmath}

\title{Física II}
\date{}
\author{Andrea Velásquez}

\begin{document}

\maketitle

\section*{Termodinámica}
La termodinámica es el estudio de las variaciones en temperatura y los cambios de estado de cualquier sistema debido al intercambio de energía con su entorno.
\subsection*{Gases ideales}
\begin{itemize}
\item Están compuestos por moléculas puntuales que no interactúan entre ellas.
\item La distancia entre moléculas es grande
\item Tienen baja densidad
\item Se rigen por medio de la siguiente ecuación:

\begin{equation}\label{eq:1.1}
\ PV = nRT
\end{equation}
en la que R es la constante de gas ideal. \\\\

\item Si la masa del gas se mantiene constante, también su número de moles, $n$, se mantendrá constante, y así 
\begin{equation}
\dfrac{PV}{T} = nR = c
\end{equation}
donde c es alguna constante, y por consiguiente: 
\begin{equation}
\dfrac{P_iV_i}{T_i} = \dfrac{P_fV_f}{T_f}
\end{equation}
\item $PV = nRT$ es una ecuación con aproximación, porque no se está considerando el tamaño de las moléculas ni la interacción entre ellas. 
\item La ecuación corregida sería la de Van der Waals:
\begin{equation}
\ (P+\dfrac{an^2}{V})(V-nb) = nRT
\end{equation}
\item ¿De qué depende la presión y la temperatura? 
\item Asumimimos:
\subitem $V = $ constante; paredes rígidas
\subitem moléculas puntuales (tamaño = 0)
\subitem no hay interacción entre moléculas
\subitem las colisiones son elásticas
\subitem movimiento en direcciones aleatorias
\item ¿Qué fuerza siente la pared cuando hay muchas moléculas golpeándola en un tiempo d$t$?
\begin{equation}
\vec{F} = \dfrac{d\vec{p}}{dt}
\end{equation}
\begin{footnotesize}
\begin{equation}
\ d\vec{p_y}= 0  
\end{equation}
\begin{equation}
 d\vec{p_x}= m\vec{v_x}-m(-\vec{v_x})= 2m\vec{v_x}
\end{equation}
\end{footnotesize}
\item El número de moléculas en un cilindro está dado por la ecuación:
\begin{equation}
\dfrac{N}{Vol}
\end{equation}
\item El número de moléculas golpeando en un área $A$ y un tiempo d$t$ está dado por la ecuación:
\begin{equation}
\dfrac{1}{2}\dfrac{N}{V}Adt
\end{equation}
\item Por consiguiente,
\begin{equation}
d\vec{p_x} = 2m\vec{v_x}(\dfrac{1}{2}\dfrac{N}{V}Adt)
\end{equation}
\begin{equation}
\vec{F_x} = \dfrac{d\vec{p_x}}{dt} = m\vec{v_x}\dfrac{N}{V}A
\end{equation}
se usa $V_x$ promedio porque no todas las moléculas tienen la misma velocidad en $x$ $\vec{v_x}$.
\item Como el movimiento es aleatorio, en todas las direcciones el movimientode las moléculasen promedio se ve igual.
\end{itemize}

\end{document}
