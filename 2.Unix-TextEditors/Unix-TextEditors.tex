\documentclass[12pt]{article}
\usepackage[margin=0.5in]{geometry}
\usepackage{amsmath, amsthm}
\usepackage[spanish] {babel}


\title{\textsc{Editores de texto en UNIX}}

\begin{document}

\date{}
\maketitle

Un editor de texto es el lugar en el cual vamos a escribir codigo, luego este va a ser interpretado 
por la m\'aquina via un compilador.\\

Algunos de los editores de texto mas comunes en UNIX son:

\begin{itemize}
\item emacs
\item vi 
\item gedit
\item pico 
\item nano 
\end{itemize}

En nuestra clase vamos a poner especial atenci\'on a emacs.

\begin{center}
\section*{Emacs}
\end{center}

Ejecutar emacs:\\

\texttt{emacs} \verb"archivo &"  \\

Esto me abrira una ventana fuera de la terminal. El $\&$ se usa para que la temrinal siga activa.\\

\texttt{emacs} \verb"archivo &" \texttt{-nw}  \\

El \texttt{-nw} (no window) abrira emacs dentro de la terminal.\\

Cortar texto:\\

\texttt{Ctrl+w}\\

Copiar texto:\\

\texttt{M+w}

Pegar texto:\\

\texttt{Ctrl+y}\\


Guardar:\\

\texttt{Ctrl+x+s}\\

Salir de emacs:\\

\texttt{Ctrl+x+c}\\

\textsc{Sitio oficial de emacs:}\\
\verb"https://www.gnu.org/software/emacs/"\\

\textsc{Manuales y tutorailes:}\\
\verb"https://www.gnu.org/software/emacs/manual/html_node/emacs/index.html"\\
\verb"http://www.drpaulcarter.com/cs/emacs.php"\\

\end{document}