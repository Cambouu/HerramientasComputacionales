\documentclass[12pt]{article}
\usepackage[margin=0.5in]{geometry}
\usepackage{amsmath, amsthm}
\usepackage[spanish] {babel}
\usepackage{hyperref} 


\title{\textsc{Editores de texto en UNIX \& Scripts} }

\begin{document}

\date{}
\maketitle

Un editor de texto es el lugar en el cual vamos a escribir codigo, luego este va a ser interpretado 
por la m\'aquina via un compilador.\\

Algunos de los editores de texto mas comunes en UNIX son:

\begin{itemize}
\item emacs
\item vi / vim (escribiendo vimtutor en la terminal hay un buen tutorial)
\item gedit
\item pico 
\item nano 
\item SublimeText3 \footnote{Acá un tutorial para instalar SublimeText3 en Linux:\newline \url{http://linuxg.net/how-to-install-sublime-text-3-build-3065-on-the-most-popular-linux-systems/}}
\end{itemize}

En nuestra clase vamos a poner especial atenci\'on a emacs.

\begin{center}
\section*{Emacs}
\end{center}

Ejecutar emacs:\\

\texttt{emacs} \verb"archivo &"  \\

Esto me abrira una ventana fuera de la terminal. El $\&$ se usa para que la temrinal siga activa.\\

\texttt{emacs} \verb"archivo &" \texttt{-nw}  \\

El \texttt{-nw} (no window) abrira emacs dentro de la terminal.\\

Cortar texto:\\

\texttt{Ctrl+w}\\

Copiar texto:\\

\texttt{M+w}

Pegar texto:\\

\texttt{Ctrl+y}\\


Guardar:\\

\texttt{Ctrl+x+s}\\

Salir de emacs:\\

\texttt{Ctrl+x+c}\\

\textsc{Sitio oficial de emacs:}\\
\verb"https://www.gnu.org/software/emacs/"\\

\textsc{Manuales y tutorailes:}\\
\verb"https://www.gnu.org/software/emacs/manual/html_node/emacs/index.html"\\
\verb"http://www.drpaulcarter.com/cs/emacs.php"\\


\section*{Scripts:}

Los scripts son codigos que cumplen ciertas funciones dentro de UNIX. Por ejemplo
un script puede ser dise\~ado para bajar un archivo. \\


Primero abramos un editor de texto donde escribir nuestro script:\\ 

\begin{verbatim}

emacs downloading.sh

\end{verbatim}

Luego una vez abierto \verb"emacs" escribimos el codigo para bajar archivos de internet, en este
caso vamos a bajar el libro de la Metamorfosis:\\

\verb"wget"  \href{https://raw.githubusercontent.com/jngaravitoc/HerramientasComputacionales/master/Lectures/2.Unix-TextEditors/Hands-on/metamorphosis.txt}{metamorphosis.txt} \\

Ahora ejecutemos el Script:\\

\verb"chmod u+x downloading.sh" \#  Esto vuelve mi script ejecutable (Por eso ahora es color verde).\\

\verb"./downloading.sh"   \# Ejecuta el script




\end{document}