\documentclass[12pt]{article}
\usepackage{enumerate}
\usepackage[hmargin=2.0cm,vmargin=1cm]{geometry}


\title{\begin{LARGE}
{Hands On: Consola y comandos b\'asicos en Linux}
\end{LARGE}}
\begin{document}

\maketitle
\textbf{Este ejercicio no se debe entregar, es solo para practicar lo visto en clase.}\\
\\
0. Mirar donde estamos parados dentro del computador. \\
\\
1. Crear una carpeta que se llame \verb"Herraminetas Computacionales" dentro de esta carpeta
organizaremos todo el contenido del curso a medida que vayamos avanzando.\\
\\
2. Bajar la carpeta \verb"Ejercicio.tar" de \\ \verb"https://github.com/jngaravitoc/HerramientasComputacionales/blob/master/Lectures/"\\
\verb"1.Unix-Console/Hands-on/Ejericicio.tar"\\
\\
3. Descomprimir la carpeta y mirar el tamaño de toda la carpeta descomprimida.\\
\\
4. Mirar que hay dentro de la carpeta.\\
\\
5. Organizar la carpeta de manera que en \verb"Clases\" queden las 14 Clases, en \verb"Tareas\"
las 12 Tareas y en \verb"Ejericios\" los 12 Ejercicios. \\
\\
6. Dentro de la carpeta Ejercicios bajar el archivo \textbf{data.txt} de \verb"dar url"\\
\\
7. Cuantas lineas, palabras y caracteres hay en el archivo?\\
\\
8. Cuanto pesa el archivo?	\\
\\
9. Cambiar el nombre del archivo.\\
\\
10. Cuantas veces aparece el numero 30 en el archivo?\\
\\
11. Borrar el archivo data.txt\\

\end{document}