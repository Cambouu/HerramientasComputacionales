\documentclass[12pt]{article}
\usepackage{enumerate}
\usepackage[hmargin=2.0cm,vmargin=1cm]{geometry}
\usepackage[utf8]{inputenc}
\usepackage{graphicx}
\usepackage{float}
\usepackage{cite}
\usepackage{natbib}

\title{\begin{LARGE}
{Cookbook Clase 1}
\end{LARGE}}

\begin{document}

\maketitle

En este cookbook encontrar\'an paso a paso lo que realizaremos en clase.

\section{Comandos basicos de UNIX}

A continuaci\'on veremos los comandos basicos para trabajar en \verb+UNIX+.
Para saber mas informaci\'on sobre cada uno de los comandos podemos ver
el manuel escribiendo en la terminal:\\

\verb+$ man comando+ \\

Para salir del manual usamos la letra \verb+q+

\subsection{Directorios:}

Para saber en que directorio estamos escribimos: \\

\verb+$ pwd+ En mi caso imprimira:\\

\verb+$/home/t430+ \\

Debemos estar en \verb+home/username+ si no estan ah\'i usen \verb+$ cd+. una vez ah\'i listemos los 
archivos y directorios que hay usando \verb+ls+\\ 

Para esta clase entraremos al directorio \verb+Herramientas_Computacionales\clase2+ \\
\verb+cd Herramientas_Computacionales\clase2+\\
En caso de que no esten estos directorios los creamos con \verb+mkdir+



\subsection{Archivos:}

Para esta actividad debemos descargar el libro \verb+Sainte Beuve+ de Baudelaire\\

\verb+$ wget https://raw.githubusercontent.com/jngaravitoc/HerramientasComputacionales/+\\
\verb+master/Lectures/2.Unix-TextEditors/Hands-on/Sainte-Beuve.txt+

Para explorar un poco este libro podemos usar \verb+less+ as\'i \verb+less Sainte-Beauve.txt+ para salir de explorar el archivo usamos \verb+q+.\\

Si usamos el "flag" \verb+ls -l+ podemos ver los permisos del archivo y la ultima fecha de modificaci\'on.\\
Si queremos cambiar esta fecha de modificaci\'on podemos usar \verb+touch Sainte-Beuve.txt+

Para ver las primeras lineas de este libro usamos \verb+head+ predeterminado se muestran las primeras 10 lineas, 
para cambiar esto usamos \verb+head -20 Sainte.Beuve.txt+ esto mostrara las primeras 20 lineas. Analogamente 
\verb+tail+ muestra las ultimas 10 lineas.\\

Para redirigir el \verb+output+ que se imprime en la consola a un archivo se usa \verb+>+ por ejemplo si queremos
hacer un archivo con las primeras 100 lineas del libro Sainte-Beuve podemos hacerlo as\'i \verb+head -100 Sainte-Beuve.txt > Sainte-Beuve100.txt+.\\

Para insertar mas lineas \textbf{debajo} de las 100 lineas que ya creamos usamos \verb+>>+ en vez de \verb+>+\\ 

\textbf{Ejercicio:} Al archivo Sainte-Beuve100.txt agreguele las siguientes 200 lineas y nombrelo Sainte-Beuve200.txt.\\

Ayuda: Si necesita renombrar archivos use \verb+mv+ si necesita copiar un arvhivo use \verb+cp+ para saber la sintaxis 
de estos comandos veal el manual de estos.\\

Una forma alteratva de concatenar archivos es usando \verb+cat+:\\
\verb+cat Sainte-Beuve100.txt Sainte-Beuve200.txt+  hagan lo mismo usando \verb+tac+ '?que sucede?.\\

Como estamos seguros de que \verb+Sainte-Beuve200.txt+ tiene 200 lineas?, Para esto usamos \verb+wc+ lean 
el manual y respondan las siguientes preguntas:\\

1. ?'Cuantas lineas, paabras y bytes hay en \verb+Saint-Beuve.txt+?

\subsection{Explorando archivos:}

Muchas veces necesitamos modificar y manipular archivos, en general se\'ran datos de experimentos o simulaciones que hayamos realizado.
Por ejemplo: Como pudieramos saber cual es la vocal que mas se repite en Saint-Beuve.txt? 


\verb+grep + se usa para encontrar patrones en archivos, este imprime las lineas en las cuales aparece el patron que estamos buscando.
%sed
Si queremos buscar cuantas veces aparece la vocal \verb+a+ lo hariamos as\'i \verb+grep a Saint-Beuve.txt+. Hasta el momento hemos manipulado 
el texto pero no modificado, imaginense ahora que por alguna extra\~na raz\'on la editorial requiere que todos los espacios sean reemplazados 
por comas \verb+,+, como podr\'iamos hacer esto?  


\verb+$ sed 's\ \ ,\g'+ Donde \verb+s+ es para 

%awk

\verb++

%sort

%\subsection{Compresion de arvhivos.}

\section{Scripts}

Un script es un archivo en el cual se escriben una secuencia de comandos
que tienen como fin realizar una tarea que el usuario desee. Cuando se ejecute 
el script los comandos se iran ejecutando seg\'un el orden en el que escribimos los comandos. Por ejemplo hagamos un script que baje un archivo de datos y arroje el peso en bytes del archivo.\\

1. Abrir un editor de texto un archivo que tenga una extension $.sh$ en este 
caso usaremos $emacs$ como nuestro editor de texto. Para abir un archivo en emacs escribimos en la consola:\\
\begin{verbatim}
emacs script-1.sh &
\end{verbatim}
Esto abrir\'a una ventana fuera de la terminal en la cual escribiremos nuestros comandos.

2. Ahora escribamos el comando que bajar\'a los datos.

\begin{verbatim}
wget https://raw.githubusercontent.com/forero/ComputationalLab/master/2015-1/Hands-on/T1/data.tar.gz
\end{verbatim}

Para pegar en \verb+Emacs+ se utiliza \verb|ctrl+y| y pra guardar los cambios usamos \verb"ctrl+x+s"\\

3. En la terminal ejecutamos el script \verb+script-1.sh+, para volver ejecutable este archivo debemos cambiar los permisos 
del archivo usando el comando \verb+chmod+ \\

\begin{verbatim}
chmod u+x script-1.sh
\end{verbatim}


Para ejecutar el archivo escribimos \verb|./script-1.sh|

4. Volvemos a la ventana de \verb+Emacs+ y escribimos el comando descomprimir los datos y para ver el peso en bytes del archivo.

\begin{verbatim}
tar -xzvf data.tar.gz
du -b data
\end{verbatim}

5. Guardamos y volvemos a ejecutar.

\end{document}
