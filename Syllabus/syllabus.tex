\documentclass[letterpaper]{article}
\usepackage{enumerate}
\usepackage[hmargin=3.0cm,vmargin=2.25cm]{geometry}

\begin{document}

\begin{center}
\textsc{\LARGE Herramientas Computacionales}\\
\textsc{\large Unversidad de los Andes.}\\
\end{center}
\large{FISI-2026, Secci\'on 3, Semestre 2014-20.}\\
\large{Profesor: Juan Nicol\'as Garavito Camargo.}\\
\large{Email: jn.garavito57}\\
\large{Sal\'on: LL204}\\
\large{Martes: 7:00-8:30}\\
\large{Horario de Atenci\'on: Viernes 14:00 - 16:00}\\
\large{Pagina del curso: \verb"www.github.com/jngaravitoc/HerramientasComputacionales"}


\section*{Introduci\'on:}

La ciencia a evolucionando de tal forma que el uso de los computadores
es indispensable para hacer investigaci\'on. La cantidad de datos que se
obtienen d\'ia a d\'ia necesitan de una capacidad computacional adecuada
para manipularlos y deducir informaci\'on de estos, que ser\'a luego utilizada para
 realizar o comparar con modelos.


\section*{Objetivos:}

El objetivo principal del curso es:

\begin{itemize}
\item Dar las principales herramientas y conceptos computacionales necesarios
los cuales facilitaran el trabajo diario de los estudiantes en sus estudios y
en su futuras investigaciones.
\end{itemize}

\section*{Competencias a desarrollar:}

Al finalizar el curso, se espera que el estudiante este en capacidad de:

\begin{itemize}
\item Utilizar computadores con sistema operativo UNIX.
\item Presentar documentos en \LaTeX.
\item Manipular, analizar y visualizar datos.


\end{itemize}

\section*{Metodolog\'ia:}

Cada sesi\'on ser\'a te\'orico-practica, en la primera parte de la clase $\sim 40$ min el profesor
dar\'a una introducci\'on del tema, en algunas clases esto se har\'a interactivamente en el computador.
En la segunda parte de la clase se realizar\'a un taller con el fin de practicar
lo visto en clase.

La pagina (repositorio) del curso se actualiza permanentemente con nuevo material, este est\'a organizado en las siguientes carpetas:\\

\begin{itemize}
\item \textbf{Syllabus/:} Contiene el programa del curso (i.e: Este pdf).
\item \textbf{Lectures/:} Contiene las notas de cada clase (Presentaciones y/o IPython-notebook).
\item \textbf{Grades/:} Contiene las notas de cada taller.
\end{itemize}



\section*{Programa:}
-\ Semana 1 (Julio 29): [Linux] Comandos b\'asicos de UNIX. \\
\\
-\ Semana 2 (Agosto 5): [Linux] Editores de texto (Emacs). \\
\\
-\ Semana 3 (Agosto 12): [\LaTeX] L\'ogica de compilaci\'on (Documentclass article, Secciones, Ecuaciones).\\
\\
-\ Semana 4 (Agosto 19): [\LaTeX] Tablas, Figuras y Bibliograf\'ia. \\
\\
-\ Semana 5 (Agosto 26)*: [Python] Presentaci\'on de Python, Iteraci\'on. \\
\\
-\ Semana 6 (Septiembre 2): [Python] Recursividad y Descomposici\'on en funciones. \\
\\
-\ Semana 7 (Septiembre 9): [Python] Visuzalizaci\'on de datos (Matplotlib). \\
\\
-\ Semana 8 (Septiembre 16): Encontrar ra\'ices: M\'etodos de bisecci\'on y  M\'etodo de \indent Newton/Rhapson. \\
\\
-\ Semana 9 (Septiembre 23): \textbf{Semana de trabajo individual.}
\\
-\ Semana 10 (Septiembre 30): Histogramas y Distribuci\'on Normal. \\
\\
-\ Semana 11 (Octubre 7): Valor medio y dispersi\'on como mejor estimado e incertidumbre. \\
\\
-\ Semana 12 (Octubre 14): Regresiones lineales y Ajuste de m\'inimos cuadrados. \\
\\
-\ Semana 13 (Octubre 21): Distribuci\'on de Poisson, Binomial. \\
\\
-\ Semana 14 (Octubre 28): Modelos computacionales sencillos. Simulaci\'on de marcha aleatoria. \\
\\
-\ Semana 15 (Noviembre 4): Simulaciones Montecarlo.  Estimaci\'on del n\'umero $\pi$. \\
\\
-\ Semana 16 (Noviembre 11): Bono.\\


\section*{Evaluaci\'on:}

En total se entregan 12 talleres, de los cuales se quita la peor y la mejor nota.
Por lo tanto se califican 10 talleres, el promedio de estos da la nota final del curso. Es decir que
cada taller tiene un valor del $10\%$ de la nota final.

\section*{Bibliograf\'ia:}
\begin{itemize}
\item Guttag, John V. (2013). \textit{Introduction to Computational and Programming Using Python}, The MIT Press.
\item \verb"http://www.codeacademy.com/"
\item Lagtangen, H.P. A Primer on Scientific Programming with Python. 1-718 (Springer, 2009).
\item Lee, K.D. Python Programming Fundamentals. 1-243(Springer, 2011).
\item van, Vugt, S. Beginning the Linux Command Line. 1-381 (Apress, 2009).
\item Gr$\ddot \rm{a}$tzer, G. More Math Into Latex. 1-269 (Springer, 2007).
% \item Coursera
\end{itemize}

\end{document}
