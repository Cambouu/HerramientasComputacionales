\documentclass[letterpaper]{article}
\usepackage{enumerate}
\usepackage[hmargin=3.0cm,vmargin=2.25cm]{geometry}

\begin{document}

\begin{center}
\textsc{\LARGE Herramientas Computacionales}\\
\textsc{\large Unversidad de los Andes.}\\
\end{center}
\large{FISI-2026, Semestre 2014-20.}\\
\large{Profesor: Juan Nicol\'as Garavito Camargo.}\\
\large{Email: jn.garavito57}\\
\large{Sal\'on:}\\
\large{Viernes: 7:00-8:30}\\
\large{Horario de Atenci\'on: Martes 14:00 - 17:00}\\
\large{Pagina del curso: \verb"www.github.com/jngaravitoc/HerramientasComputacionales"}

%\section*{Objetivo:}

%Dar a los estudiantes las herramientas computacionales necesarias , tales como:
%\LaTeX, Python , IPython, IPython-notebook 

%poner alguna grafica pertinente.

\section*{Metodolog\'ia:}

Cada sesi\'on ser\'a te\'orico-practica, en la primera parte de la clase $\sim 40$ min el profesor 
dar\'a una introducci\'on del tema, en algunas clases esto se har\'a interactivamente en el computador. 
En la segunda parte de la clase se realizar\'a un taller con el fin de practicar
lo visto en clase.

La pagina (repositorio) del curso se actualiza permanentemente con nuevo material, este est\'a organizado en las siguientes carpetas:\\

\begin{itemize}
\item \textbf{Syllabus/:} Contiene el programa del curso (i.e: Este pdf).
\item \textbf{Lectures/:} Contiene las notas de cada clase (Presentaciones y/o IPython-notebook).
\item \textbf{Grades/:} Contiene las notas de cada taller.
\end{itemize}
 
 

\section*{Programa:}
-\ Semana 1 (Agosto 1): [Linux] Comandos b\'asicos de UNIX. \\
\\
-\ Semana 2 (Agosto 8): [Linux] Editores de texto (Emacs). \\
\\
-\ Semana 3 (Agosto 15): [\LaTeX] L\'ogica de compilaci\'on (Documentclass article, Secciones, Ecuaciones).\\
\\
-\ Semana 4 (Agosto 22)*: [\LaTeX] Tablas, Figuras y Bibliograf\'ia. \\
\\
-\ Semana 5 (Agosto 29)*: [Python] Presentaci\'on de Python, Iteraci\'on. \\
\\
-\ Semana 6 (Septiembre 5): [Python] Recursividad y Descomposici\'on en funciones. \\
\\
-\ Semana 7 (Septiembre 12): [Python] Visuzalizaci\'on de datos (Matplotlib). \\
\\
-\ Semana 8 (Septiembre 19): Encontrar ra\'ices: M\'etodos de bisecci\'on y  M\'etodo de \indent Newton/Rhapson. \\
\\
-\ Semana 9 (Septiembre 26): Histogramas y Distribuci\'on Normal. \\
\\
-\ Semana 10 (Octubre 3): Valor medio y dispersi\'on como mejor estimado e incertidumbre. \\
\\
-\ Semana 11 (Octubre 10): Regresiones lineales y Ajuste de m\'inimos cuadrados. \\
\\
-\ Semana 12 (Octubre 17): Distribuci\'on de Poisson, Binomial. \\
\\
-\ Semana 13 (Octubre 24): Modelos computacionales sencillos. Simulaci\'on de marcha aleatoria. \\
\\
-\ Semana 14 (Octubre 31): Simulaciones Montecarlo.  Estimaci\'on del n\'umero $\pi$. \\


\section*{Evaluaci\'on:}

En total se entregan 12 talleres, de los cuales se quita la peor y la mejor nota. 
Por lo tanto se califican 10 talleres el promedio de estos da la nota final del curso. Es decir que 
cada taller tiene un valor del $10\%$ de la nota final.

\section*{Bibliograf\'ia:}
\begin{itemize}
\item Guttag, John V. (2013). \textit{Introduction to Computational and Programming Using Python}, The MIT Press.
\item \verb"http://www.codeacademy.com/"
% \item Coursera
\end{itemize}

\end{document}
